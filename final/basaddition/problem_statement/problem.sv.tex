\problemname{Basaddition}

\begin{figure}
\centering
\begin{verbatim}
 37A
+A50
----
131A
\end{verbatim}
\end{figure}

I figuren ses en addition av två tal -- men i vilken bas är talen uttryckta och additionen utförd?

Uppgifen består i att skriva ett program som tar emot \emph{två termer} och \emph{en summa} i form av tre strängar. Programmet ska sedan bestämma i vilka baser additionen är korrekt.

Testen ska omfatta de femton baserna 2 till 16. För att beteckna talen 10, 11, 12, 13, 14 och 15 används versalerna A, B, C, D, E och F. Det kan förutsättas att det alltid finns minst en bas för vilken additionen är korrekt.

\section*{Indata}
Indata består av tre strängar, var och en med maximalt 10 tecken, på tre rader efter varandra. De två första termerna betecknar termerna i additionen, och den tredje raden summan.

\section*{Utdata}
Utdatan ska bestå av ett antal heltal: alla baser som additionen är korrekt i. Baserna ska anges sorterade, i stigande ordning.


\section*{Poängsättning}
Din lösning kommer att testas på flera testfall. För att få 100 poäng så måste du klara alla testfall.

