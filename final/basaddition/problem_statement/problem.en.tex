\problemname{Base Addition}

\begin{figure}
\centering
\begin{verbatim}
 37A
+A50
----
131A
\end{verbatim}
\end{figure}

In the figure, an addition of two numbers is shown -- but in which base are the numbers written, and
in which base is the addition performed?

The task is to write a program that receives \emph{two terms} and \emph{a sum} in the form
of three strings. The program should then determine in which bases the addition is correct.

The tests will cover the fifteen bases from 2 to 16. To represent the numbers $10$, $11$, $12$, $13$, $14$, and $15$, 
the capital letters \texttt{A}, \texttt{B}, \texttt{C}, \texttt{D}, \texttt{E}, and \texttt{F} are used. 
It can be assumed that there is always at least one base for which the addition is correct.

\section*{Input}
The input consists of three strings, each with at most $10$ characters, on three consecutive lines. 
The first two strings represent the terms of the addition, and the third line represents the sum.

\section*{Output}
The output should consist of a number of integers: all bases for which the addition is correct. 
The bases should be listed in ascending order.

\section*{Scoring}
To receive 100 points, your solution must pass all test cases.
